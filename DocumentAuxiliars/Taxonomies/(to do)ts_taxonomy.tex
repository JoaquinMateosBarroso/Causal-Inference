\documentclass{article}
\usepackage[utf8]{inputenc}

\usepackage[a4paper,margin=2cm]{geometry}

\usepackage{graphicx}
\usepackage{tikz}
\usepackage{forest}
\usetikzlibrary{trees,positioning,shapes,shadows,arrows.meta}


\begin{document}

\tikzset{
  basic/.style  = {draw, text width=2cm, drop shadow, font=\sffamily, rectangle},
  root/.style   = {basic, rounded corners=2pt, thin, align=center, fill=white},
  level-2/.style = {basic, rounded corners=6pt, thin,align=center, fill=white, text width=3cm},
  level-3/.style = {basic, thin, align=center, fill=white, text width=1.8cm}
}

\begin{figure}
    \centering
\begin{tikzpicture}[
  level 1/.style={sibling distance=12em, level distance=5em},
%   {edge from parent fork down},
  edge from parent/.style={->,solid,black,thick,sloped,draw}, 
  edge from parent path={(\tikzparentnode.south) -- (\tikzchildnode.north)},
  >=latex, node distance=1.2cm, edge from parent fork down]

% root of the the initial tree, level 1
\node[root] {\textbf{Taxonomy}}
% The first level, as children of the initial tree
  child {node[level-2] (c1) {\textbf{Category 1}}}
  child {node[level-2] (c2) {\textbf{Category 2}}}
  child {node[level-2] (c3) {\textbf{Category 3}}}
  child {node[level-2] (c4) {\textbf{Category 4}}};

% The second level, relatively positioned nodes
\begin{scope}[every node/.style={level-3}]
\node [below of = c1, xshift=10pt] (c11) {item 1-1};
\node [below of = c11] (c12) {item 1-2};
\node [below of = c12] (c13) {item 1-3};

\node [below of = c2, xshift=10pt] (c21) {item 2-1};
\node [below of = c21] (c22) {item 2-2};
\node [below of = c22] (c23) {item 2-3};
\node [below of = c23] (c24) {item 2-4};

\node [below of = c3, xshift=10pt] (c31) {item 3-1};
\node [below of = c31] (c32) {item 3-2};
\node [below of = c32] (c33) {item 3-3};
\node [below of = c33] (c34) {item 3-4};
\node [below of = c34] (c35) {item 3-5};

\node [below of = c4, xshift=10pt] (c41) {item 4-1};
\node [below of = c41] (c42) {item 4-2};
\end{scope}

% lines from each level 1 node to every one of its "children"
\foreach \value in {1,2,3}
  \draw[->] (c1.195) |- (c1\value.west);

\foreach \value in {1,...,4}
  \draw[->] (c2.195) |- (c2\value.west);

\foreach \value in {1,...,5}
  \draw[->] (c3.195) |- (c3\value.west);
  
\foreach \value in {1,2}
  \draw[->] (c4.195) |- (c4\value.west);
\end{tikzpicture}
    \caption{This is a simple Taxonomy}
    \label{fig:my_label}
\end{figure}



\end{document}
